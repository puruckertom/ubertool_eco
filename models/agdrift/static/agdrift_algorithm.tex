\documentclass[10pt]{article}
\usepackage{graphicx}
\usepackage{endfloat}
\usepackage{amssymb}
\usepackage{amsmath}
\usepackage{setspace}
\usepackage[scientific-notation=true]{siunitx}
\usepackage[margin = 1in]{geometry}
\hoffset = -1.3in %set margin offset horizontally for the document
\allowdisplaybreaks
\begin{document}
\begin{align*}
& \hspace{-0.75in}  \text{Tier I Ground Sprayer Assessment Methodology for low boom height} \nonumber \\  & \hspace{-0.75in} \text{and orchard/airblast scenarios:}\nonumber \\
& \hspace{-0.75in}  \text{(Based on SDTF field trials)} \nonumber \\
D_{x} & = \frac{c}{\left(1+ax\right)^{b}} \nonumber \\
&    \hspace{-.75in}  \text{Curve shape parameters were estimated using least-squares analysis.} \nonumber \\
&    \hspace{-.75in}  \text{parameters:} \nonumber \\
D_{x} & = \text{Deposition level relative to the nominal application rate} \nonumber \\
x & = \text{downwind distance} \nonumber \\
a,b,c & = \text{curve shape parameters} \nonumber \\
& \hspace{-0.75in}  \text{A high boom model was developed by extending the low boom model.} \nonumber \\
& \hspace{-0.75in}  \text{Ground sprayer assessment methodology for high boom height:} \nonumber \\
D_{x} & = \frac{c}{\left(1+ax\right)^{b}}\left(1 + A * exp\left( -Bx\right)\right) \nonumber \\
D_{x} & = \text{Deposition level relative to the nominal application rate} \nonumber \\
x & = \text{downwind distance} \nonumber \\
a,b,c & = \text{curve shape parameters} \nonumber \\
A, B & = \text{determined by matching the high boom data with at 25 ft} \nonumber \\ &\hspace{1.3em} \text{and assuming that high boom deposition is ten percent} \nonumber\\&\hspace{1.3em}\text{higher than low boom deposition at 2600 ft} \nonumber \\
\end{align*} 
\end{document}